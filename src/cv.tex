\documentclass[7pt]{article}

\usepackage{variables}

\usepackage{fontspec}
\usepackage{hyperref}
\usepackage{geometry}
\usepackage[none]{hyphenat}
\usepackage{longtable}

\def\middot{\textperiodcentered~}

\hypersetup{
  colorlinks = true,
  urlcolor = black,
  pdfauthor = {\name: \name},
  pdfkeywords = {David, Richards, software, developer, Brighton},
  pdftitle = {\name: Curriculum Vitae},
  pdfsubject = {Curriculum Vitae},
  pdfpagemode = UseNone
}

\geometry{
 a4paper,
 total={170mm,257mm},
 left=5mm,
 right=15mm,
 top=10mm,
 bottom=10mm,
}

\setmainfont{Lato}
\pagestyle{empty}

\setlength\parindent{0em}

\renewcommand{\labelitemii}{$\bullet$}

\nopagebreak[4]

\begin{document}

\centerline{\huge \bf \name}

\vspace{0.25in}

\begin{minipage}{1.0\textwidth}
  \centering
  \address
  \break
  \href{mailto:\email}{\email}
  \middot \href{tel:\phone}{\phone}
  \break
  \href{https://dcrichards.com/}{dcrichards.com}
  \middot \href{https://github.com/DCRichards}{github.com/DCRichards} 
\end{minipage}

\vspace{0.25in}

\begin{tabular}{@{} p{0.15\textwidth} p{0.85\textwidth}}
  \textbf{EDUCATION} & \begin{minipage} [t] {0.85\textwidth} 2011 - 2015 \href{https://sussex.ac.uk/}{\textbf{University of Sussex}}, Brighton \\
  \textbf{First Class}: BSc (Hons) Computer Science (with an Industrial Placement Year) \\
  \newline
    Final Year Project: Two-factor Authentication in Wearable Technologies, developing a two-factor authentication application for the Pebble Smartwatch and Google Glass using Time-Based One-Time Passwords (TOTP). \\
    \newline
    Notable Modules:
    \begin{itemize}
      \setlength\itemsep{-4pt}
      \item Web Applications \& Services (J2EE, JSF, GlassFish, Java DB, AWS)
      \item Databases (SQL)
      \item Machine Learning, Natural Language Engineering (Python)
      \item Comparative Programming (Haskell, Prolog)
      \item Compilers \& Computer Architecture (MIPS, x86)
    \end{itemize}
    Additional Experience:
    \begin{itemize}
      \item Student Life speaker for Informatics Applications Visit Days, giving talks on course content and experience to prospective students and parents.
    \end{itemize}
  \end{minipage} \\
  & \\
  & 2009-2011 South Downs College, Waterlooville \\
  & A Level: Computing (A), Mathematics (B), Chemistry (B) \\
  & \\
  & 2004-2009 Park Community School, Havant \\
  & GCSE: 11 subjects A* to B grade including Maths (A*) and English (A) \\
\end{tabular}

\vspace{0.25in}

\begin{longtable}{@{} p{0.15\textwidth} p{0.85\textwidth}}
  \textbf{PROFESSIONAL EXPERIENCE} & \begin{minipage} [t] {0.85\textwidth}
    February 2016 - Present \href{https://ocasta.com}{\textbf{Ocasta}, Brighton} \middot Software Developer \\
    \newline
    Ocasta specialise in apps, web and everything in between. Ocasta are also the creators of the Oplift customer engagement platform.
    \begin{itemize}
      \setlength\itemsep{-1pt}
      \item Android development on a number widely-used of applications, including Virgin Media WiFi, providing WiFi Hotspot access to over 500,000 customers. Making use of modern Android tooling, including Kotlin/Java with RxJava, Retrofit 2, Robolectric, Espresso, Room and Android Architecture/Lifecycle Components.
      \item Technical lead on the Home WiFi project, extending Virgin Media WiFi to provide customers with direct control of their router. This included building an abstracted REST API on top of the SNMP over REST API provided.
      \item Developing services and microservices in Node.js (Hapi.js/Koa, Mocha/Chai, JSHint/Eslint, MongoDB) and Go.
      \item Building GraphQL APIs and integrating these into existing RESTful architecture as well as part of standalone microservices.
      \item Wordpress development (PHP, Hack(lang)/HHVM) to provide REST API access to managed content.
      \item Collaborating closely with the design team to build UI to design specifications and provide feedback. Making use of design collaboration tools such as Sketch and Zeplin.
      \item Managing multiple agile processes including story estimation and prioritisation both within and across multiple projects.
    \end{itemize}
  \end{minipage} \\
  & \\
  & \begin{minipage} [t] {0.85\textwidth}
    January 2018 - Present \href{https://www.hiddenfield.com}{\textbf{HiddenField}} \middot Contract Developer (Remote)\\
    \newline
    HiddenField deliver cyber security products through agile solutions.
    \begin{itemize}
      \item Recent projects include a hardware wallet integration for a leading cryptocurrency organisation, building an end-to-end testing framework between device and client to ensure integrity, security and testability across all stages of the development process.
    \end{itemize}
  \end{minipage} \\
  & \\
  & \begin{minipage} [t] {0.85\textwidth}
    July 2015 - December 2015 \textbf{Mobbu}, Brighton \middot Developer \\
    \newline
    Rejoined the team at Mobbu to work on their new flagship product PassWear: A mobile and smartwatch based two-factor authentication solution.
    \begin{itemize}
      \setlength\itemsep{-1pt}
      \item Developed a large proportion of the PassWear Android SDK, including writing of the high-level API, integration with wearables and high level and API documentation.
      \item Development of wearable applications on Pebble (C), Android Wear (Java) and Apple Watch (Objective-C).
      \item Development of mobile application on both Android (Java, Gradle) and iOS (Objective-C).
      \item Continued work within the agile development team, this time using Kanban maintained by Jira and code reviews using Crucible.
      \item Managed internal dependencies using Maven and Nexus repository manager.
    \end{itemize}
  \end{minipage} \\
  & \\
  & \begin{minipage} [t] {0.85\textwidth}
    August 2013 - August 2014 \textbf{Mobbu}, Brighton \middot Intern Developer \\
    \newline
    Undertaken for my Industrial Placement Year. I joined the team during the transition from single to cross-platform mobile applications and during other web and mobile based projects.
    \begin{itemize}
      \setlength\itemsep{-1pt}
      \item Worked as part of a small Agile software development team, took part in daily stand-up meetings, retrospectives and sprint reviews and used story pointing for task estimation.
      \item Web application development in JavaScript (including using Ractive.js templating/UI library)
      \item Mobile application development using Apache Cordova on BlackBerry and Android platforms.
      \item Unit testing using QUnit/Jasmine and integration testing using CasperJS and PhantomJS.
      \item Developed RESTful Web Services in Java (JAX-RS) and Node.js including interaction with SQL Server database.
      \item Improved automation, build and deployment tools using Grunt and Ant and integrated with the Jenkins Continuous Integration server.
      \item Providing training and guidance for interns on the development processes and technology stack to ensure a smooth changeover at the end of my placement.
      \item Used Subversion and Git version control.
      \item Used Windows (IIS, Windows Server), OSX and Unix (Ubuntu) environments, including Unix command line.
    \end{itemize}
  \end{minipage}
\end{longtable}

\vspace{0.25in}

\begin{tabular}{@{} p{0.15\textwidth} p{0.85\textwidth}}
  \textbf{PERSONAL PROJECTS} & \textbf{Should I Eat There?} A simple SPA written in Vue.js, allowing users to look up the Food Hygiene rating of any UK establishment. This includes using the Vuex reactive store for the REST API integration. It can be found at \href{http://shouldieatthere.co.uk}{shouldieatthere.co.uk}. \\
  & \\
  & \textbf{Eventaroo}: Worked with a small team building an event ticketing system comprised of Node.js microservices and a Vue.js front end. This included the use of gRPC and Protobufs for inter-service communication and deployment with Kubernetes. \\
  & \\
  & I have also made a number of open-source contributions to projects such as Glide, Bintray, Robolectric (Android) and Drone CI.
\end{tabular}

\vspace{0.25in}

\begin{tabular}{@{} p{0.15\textwidth} p{0.85\textwidth}}
  \textbf{VOLUNTEER EXPERIENCE} & \textbf{Codebar Brighton (Coach)}: Codebar provides programming tuition to under-represented groups. As well as providing coaching at sessions, I recently developed and presented an interactive Git workshop, teaching Git version control to Codebar attendees. \\
  & \\
  & \textbf{CoderDojo (Mentor)}: Assisted in teaching children simple programming skills in Scratch, Python and Java. \\
  & \\
  & \textbf{Safehaven (Food Service)}: Safehaven provides a sit-down meal and support to the street community of Brighton. I provided assistance in serving food and welcoming guests.\\
  & \\
\end{tabular}

\vspace{0.25in}

\begin{tabular}{@{} p{0.15\textwidth} p{0.85\textwidth}}
  \textbf{OTHER} & Full, Clean UK Driving License
\end{tabular}

\vspace{0.25in}

\begin{tabular}{@{} p{0.15\textwidth} p{0.85\textwidth}}
  \textbf{REFERENCES} & References are available upon request.
\end{tabular}

\bigskip

\end{document}
